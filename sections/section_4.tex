\section{\S 4: Ma trận nghịch đảo của ma trận vuông}
\subsection{Định nghĩa}
Cho $A=[a_{ij}]_{n \times n}$ ma trận vuông cấp n. Nếu tồn tại ma trận B vuông cấp n thỏa: \boxed{A.B = B.A = I_N} thì B gọi là ma trận nghịch đảo của ma trận A. \\
Kí hiệu B bởi $A^{-1}$ Vậy \boxed{A.A^{-1} = A^{-1}.A = I_n} \\
*Nhận xét: \\
Lấy $\det$ hai vế:
$\det{A.A^{-1}} = \det{I_n}
\Rightarrow \det{A} \times \det{A^{-1}} = 1
\Rightarrow$ \boxed{\det{A^{-1}} = \frac{1}{\det{A}}} Tính chất đẹp 


\begin{algorithm}[H]
	\caption{Thuật toán tìm $A^{-1}$}
	\KwIn{$A=[a_{ij}]_{n \times n}$ ma trận vuông cấp n.}
	\KwOut{Ma trận nghịch đảo $A^{-1}$ nếu tồn tại, ngược lại thông báo lỗi}
	
	Tính $\det(A)$\;
	\eIf{$\det(A) = 0$}{
		\Return "Ma trận $A$ không khả nghịch."\;
	}{
		Tính nghịch đảo: \boxed{A^{-1} = \frac{1}{\det(A)} P_A}\;
		Tìm\textbf{ ma trận phụ hợp} của $A$, ký hiệu $P_A$\;
		\Return $A^{-1}$\;
	}
\end{algorithm}
\begin{tikzpicture}
	\draw[pattern=north west lines, pattern color=blue] (0,0) rectangle (1em,1em);
\end{tikzpicture}
\textbf{Cách xác định $P_A$} \\
$P_A= \begin{bmatrix}
	A_{11} & A_{12} & \cdots & A_{1n} \\
	A_{21} & A_{22} & \cdots & A_{2n} \\
	\vdots & \vdots & \ddots  & \vdots \\
	A_{n1} & A_{n2} & \cdots & A_{nn} \\
\end{bmatrix}^\intercal$ \\
Với $A_{ij} = (-1)^{i+j}.M_{ij}$ là "\textbf{phần bù đại số}" của $a_{ij} \in A$