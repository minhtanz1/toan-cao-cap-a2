\section{\S 3: Định thức của ma trận vuông}
\subsection{Định thức cấp 1}
Cho $A=[a_{11}]_{1 \times 1}$ ma  trận cấp 1: \\
Ta có: $\det{A} = |A_{11}| = a_{11}$ (= chính nó)
\subsection{Định thức cấp 2}
$\begin{bmatrix}
	a_{11} & a_{12}\\
	a_{21} & a_{22}  \\
\end{bmatrix}_{2\times 2}$ ma trận vuông cấp 2 \\
$\det{(A)} =$ 
$\begin{NiceArray}{|cc|}
	\CodeBefore [create-cell-nodes]
	\begin{tikzpicture} [shorten < = 2pt,shorten > = 2pt]
		\draw [red,->] (1-1) -- (2-2) -- ++(1.5em, -1em) node[below, scale=0.7] {\(+\)};
		\draw [blue,->] (2-1) -- (1-2) -- ++(1.5em, 1em) node[above, scale=0.7] {\(-\)} ;
	\end{tikzpicture}
	\Body
	a_{11} & a_{12}\\[2mm]
	a_{21} & a_{22}\\[2mm]
\end{NiceArray}$

\subsection{Định thức cấp 3}
$\begin{bmatrix}
	a_{11} & a_{12} & a_{13} \\
	a_{21} & a_{22} & a_{23} \\
	a_{31} & a_{32} & a_{33}
\end{bmatrix}_{3 \times 3}
=
\begin{aligned}
	&\overbrace{(-1)^{1+1} \cdot a_{11} \cdot \underbrace{\begin{vmatrix}
				a_{22} & a_{23} \\
				a_{32} & a_{33}
		\end{vmatrix}}_{M_{11}}}^{A_{11}} \\
	&+ \overbrace{(-1)^{1+2} \cdot a_{12} \cdot \underbrace{\begin{vmatrix}
				a_{21} & a_{23} \\
				a_{31} & a_{33}
		\end{vmatrix}}_{M_{12}}}^{A_{12}} \\
	&+ \overbrace{(-1)^{1+3} \cdot a_{13} \cdot \underbrace{\begin{vmatrix}
				a_{21} & a_{22} \\
				a_{31} & a_{32}
		\end{vmatrix}}_{M_{13}}}^{A_{13}}
\end{aligned}$ \\
$=a_{11}.A_{11} + a_{12}.A_{12} + a_{13}.A_{13}$ \\
*Đặt $\boxed{A_{ij}=(-1)^{ij}M_{ij}}$ gọi là \textbf{"phần bù đại số"} của phần tử $a_{ij}$\\
*$M_{11}, M_{12}, M_{13}, ... M_{ij}$ gọi là \textbf{"định thức con"} của det(A). Nó nhận từ det(A) bằng cách bỏ đi dòng i và cột j tương ứng.  \\
\begin{tikzpicture}
	\draw[pattern=north west lines, pattern color=blue] (0,0) rectangle (1em,1em);
\end{tikzpicture}
\textbf{Cần biết} \\
Cách tính trên gọi là khai triển theo dòng 1. Có thể chọn dòng hoặc cột tùy ý để khai triển. \\
\textbf{Mẹo}: Để giảm bớt tính toán thường chọn dòng hoặc cột chứa nhiều số 0 nhất để khai triển. \\
- Định thức cấp 3 tính theo sơ đồ Sarrus như sau: \\

$\begin{NiceArray}{|ccc|>{\color{gray}}c>{\color{gray}}c}
	\CodeBefore [create-cell-nodes]
	\begin{tikzpicture} [shorten < = 2pt,shorten > = 2pt]
		\draw [red,->] (1-1) -- (3-3) -- ++(1.5em, -1em) node[below, scale=0.7] {\(+\)};
		\draw [red,->] (1-2) -- (3-4) -- ++(1.5em, -1em) node[below, scale=0.7] {\(+\)};
		\draw [red,->] (1-3) -- (3-5) -- ++(1.5em, -1em) node[below, scale=0.7] {\(+\)};
		\draw [blue,->] (3-1) -- (1-3) -- ++(1.5em, 1em) node[above, scale=0.7] {\(-\)};
		\draw [blue,->] (3-2) -- (1-4) -- ++(1.5em, 1em) node[above, scale=0.7] {\(-\)};
		\draw [blue,->] (3-3) -- (1-5) -- ++(1.5em, 1em) node[above, scale=0.7] {\(-\)} ;
	\end{tikzpicture}
	\Body
	a_{11} & a_{12} & a_{13} & a_{11} & a_{12} \\[2mm]
	a_{21} & a_{22} & a_{23} & a_{21} & a_{22} \\[2mm]
	a_{31} & a_{32} & a_{33} & a_{31} & a_{32} \\
\end{NiceArray}$ \\

$\Rightarrow\det{(A) = + a_{11}a_{22}a_{33} + a_{12}a_{23}a_{31} + a_{13}a_{21}a_{32}} - \color{blue} a_{31}a_{22}a_{13} - a_{32}a_{23}a_{11} - a_{33}a_{21}a_{12}$
\subsection{Định thức cấp n}
Cho $A=(a_{ij})_{n \times n}$ ma trận vuông cấp n khi đó:\\
$\det{(A)} = a_{11}.A_{11} + a_{12}.A_{12} + \cdots + a_{1n}.A_{1n}$ (khai triển theo dòng 1)
\subsection{Tính chất định thức}
\begin{enumerate}[label=*T/c \arabic{enumi}:,leftmargin=*]
	\item $\det{(A)^\intercal} = \det{(A)}$
	\item 
	\begin{itemize}[label=-, leftmargin=*]
		\item Định thức = 0 khi có dòng và cột chứa toàn số 0
		\item Định thức = 0 khi có hai dòng hoặc hai cột giống nhau hoặc tỉ lệ nhau.
		\begin{tikzpicture}[baseline=(A.center), every node/.style={inner sep=1pt, font=\small}]
			\matrix (A) [matrix of math nodes,
			left delimiter={|},
			right delimiter={|},
			row sep=1em
			] {
				a & b & c \\
				2 & -1 & 7\\
				4 & -2 & 14\\
			};
			
			% Replace matrix B with a node displaying 0
			\node[right=.3cm of A] (B) {$=0$};
			
			% Change arrow head style by setting >=latex
			\draw[arrows = {-Stealth[]}, bend left=45]
			([xshift=-5mm] A-3-1.west)
			to node[sloped, above, pos=0.5] {\tiny {Tỉ lệ}}
			([xshift=-5mm] A-2-1.west);
		\end{tikzpicture}
		
	\end{itemize}
	\item (Nhớ) khi đổi vị trí 2 dòng hoặc 2 cột thì định thức đổi dấu.
	\begin{tikzpicture}[baseline=(C.center), every node/.style={inner sep=1pt, font=\small}]
		% First matrix named C
		\matrix (C) [matrix of math nodes,
		left delimiter={|},
		right delimiter={|},
		row sep=1em, column sep=1em] {
			a & b  \\
			x & y  \\
		};
		
		% Equals sign in its own node
		\node[right=0.5cm of C] (eq) {$=-$};
		
		% Second matrix named D
		\matrix (D) [matrix of math nodes,
		left delimiter={|},
		right delimiter={|},
		right=0.3cm of eq,
		row sep=1em, column sep=1em] {
			x & y  \\
			a & b  \\
		};
		
		% Draw a double-headed arrow on the first matrix (C)
		\draw[<->, bend left=45]
		([xshift=.5cm] C-1-2.west)
		to node[sloped, above, pos=0.5] {}
		([xshift=.5cm] C-2-2.west);
		
	\end{tikzpicture}
	\item Thừa số chung trên 1 dòng hoặc cột đưa được khỏi dấu định thức thành tích số. \\
	 $\begin{vmatrix}
		ka & kb & kc \\
		x & y & z \\
		\alpha u & \alpha v & \alpha w \\
	\end{vmatrix} = k\alpha  \begin{vmatrix}
		a & b & c \\
		x & y & z \\
		u & v & w \\
	\end{vmatrix}$
	\item Tách\\
	$\begin{vmatrix}
		a & b & c \\
		x_1 + y_1 & x_2+y_2 & x_3+y_3 \\
		u & v & w \\
	\end{vmatrix} = \begin{vmatrix}
		a & b & c \\
		x_1& x_2& c_3\\
		u & v & w \\
	\end{vmatrix} + \begin{vmatrix}
		a & b & c \\
		y_1 & y_2 & y_3 \\
		u & v & w \\
	\end{vmatrix}$
	\item Định thức không đổi khi lấy dòng này + hoặc - với k lần dòng khác (tương tự cho cột)
\end{enumerate}
\begin{tikzpicture}
	\draw[pattern=north west lines, pattern color=blue] (0,0) rectangle (1em,1em);
\end{tikzpicture}
\textbf{Tính chất đẹp} \\
\begin{enumerate}[label=\arabic{enumi}$^0$.,leftmargin=*]
	\item $\det{(AB)} = \det{(A)} \times \det{(B)}$ \\
	*Hệ quả $\det{(A^k)}= (\det{A})^k$
	\item Cho số $\alpha$ và ma trận A vuông cấp n
	\begin{itemize}
		\item $\det{\alpha A} = (\alpha)^n \times \det{A}$
		\item $\det{-A} = (-1)^n \times \det{A}$
	\end{itemize}
	\item Định thức của ma trận tam giác, đường chéo thì tích các phần tử nằm trên đường chéo chính. \\
	$\begin{vmatrix}
		a_{11} & a_{12} & a_{13} \\
		0 & a_{22} & a_{23} \\
		0 & 0 & a_{33} \\
	\end{vmatrix} = a_{11}.a_{22}.a_{33} \\
	\begin{vmatrix}
		d_{1} & 0& 0 \\
		0 & d_2 & 0 \\
		0 & 0 & d_3 \\
	\end{vmatrix} = d_1.d_2.d_3$\\
	Suy ra: $|I| = \begin{vmatrix}
		1 & & & \bigzero\\
		  & 1& & \\
		  & & \ddots& \\
		  &\bigzero & & & 1\\
	\end{vmatrix}=1$
\end{enumerate}